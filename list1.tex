\documentclass[a4paper]{article}
\usepackage[warn]{mathtext}
\usepackage{amsmath}

\usepackage[T2A]{fontenc}
\usepackage[utf8]{inputenc}
\usepackage[russian]{babel}

\usepackage[14pt]{extsizes}
\usepackage{graphicx}
\usepackage{wrapfig}
\usepackage{amssymb}
\usepackage{multicol}
\RequirePackage{caption}

\DeclareCaptionLabelSeparator{defffis}{ : }
%\DeclareMathOperator{\tg}{tg}
%\DeclareMathOperator{\dif}{\mathbf{d}}
\captionsetup{justification=centering,labelsep=defffis}

% \usepackage[unicode, pdftex]{hyperref}
\usepackage{hyperref}
\hypersetup{
  colorlinks   = true, %Colours links instead of ugly boxes
  urlcolor     = blue, %Colour for external hyperlinks
  linkcolor    = blue, %Colour of internal links
  citecolor   = red %Colour of citations
}

\usepackage{setspace,amsmath}
\usepackage[left=20mm, top=15mm, right=15mm, bottom=15mm, nohead, footskip=10mm]{geometry} % настройки полей документа
\usepackage{graphicx}

\newcommand{\norm}[1]{\left\lVert#1\right\rVert}
\newcommand{\seg}[2]{\left[#1, \:#2\right]}
\renewcommand{\ch}[0]{\mathrm{ch}\,}
\renewcommand{\sh}[0]{\mathrm{sh}\,}
\newcommand{\dx}[0]{\,dx}
\newcommand{\ds}[0]{\,ds}
\renewcommand{\phi}{\varphi}
\renewcommand{\epsilon}{\varepsilon}

\newcommand{\Deg}{^{\circ}}

\title{Листок №1}
\author{Кац Лев}

\setlength\parindent{0pt}

\begin{document}
  \maketitle

  \textbf{В данном документе несколько задач. Они упорядочены, задачи нумеруются, как влистке, и я пошлю этот документ в каждую из задач, которая здесь содержится. Это не должно как-то существенно повлиять на размер документа. Надеюсь, это тоже будет удобно.}

  \subsection*{A1 $\Diamond$ 1}
  Наше множество $A$ состоит из $n$ элементов, каждый из которых можно независимо либо взять, либо не взять (2 варианта), получая различные подмножества. Таким образом получаем $|2^A| = 2^n$

  \subsection*{A1 $\Diamond$ 2}
  $A \cap B = A \setminus (A \setminus B)$ -- элементы $A$, которые принадлежат обоим множествам; Нельзя, если мы говорим о пересечении или объединении множеств, которые были выражены путем пересечений и дополнений из $A$ и $B$, так как в результате любое такое множество будет содержать $A \cap B$, что можно доказать например индукцией по количеству членов в таком выражении.

  \subsection*{A1 $\Diamond$ 3}
  Рассмотрим сразу наиболее общий пункт -- пункт д. Итак, пусть $\sum \beta_k = n$. Пусть теперь все буквы будут разными. Тогда будет $n!$ вариантов перестановок. В нашем же случае некоторые будут повторяться. Какие? Те, в которых местами меняются местами только одинаковые буквы. Тогда ответ
  $$\frac{n!}{\prod \beta_k!} = \frac{(\sum \beta_k)!}{\prod \beta_k!},$$
  что вообще говоря не случайно является мультиномиальным коэффициентом
  $${n}\choose{\beta_1, ..., \beta_m}$$

  \subsection*{A1 $\Diamond$ 4}
  Рассмотрим сразу наиболее общий пункт -- пункт г. Какие слагаемые будут в $(a_1 + ... + a_m)^n$, если раскрыть скобки? Там будут все произвения $a_1^{\beta_1}\cdot ... \cdot a_m^{\beta_m}$, что $\beta_1 + ... + \beta_m = n, \: \beta_k \ge 0$. Если привести подобные слагаемые, какой коэффициент будет перед каждым таким слагаемым?
  Введем обозначение:
  $$\binom{n}{\beta_1, ..., \beta_m } := \frac{n!}{\prod \beta_k!}$$
  По пункту A1 $\Diamond$ 3 д это количество различных слов, полученных путем перестановок букв в слове длины $n$, где некоторые совпадают. Отождествим слово $a_1a_2...a_m$ и произведение $a_1\cdot a_2\cdot ... \cdot a_m$. У нас произведение $(a_1 + ... + a_m) \cdot ... \cdot (a_1 + ... + a_m)$. Нам нужно выбрать из $\beta_k$ скобок слагаемое $a_k$. Запишем слово, которое получится, если вместо каждой скобки записать то слагаемое, которое мы выбрали. Это будет какая-то перестановка слова $a_1^{\beta_1} ... a_m^{\beta_m}$. Более того, все эти перестановки возможны и каждая будет использоваться ровно один раз.
  Таким образом,

  $$(a_1 + ... + a_m)^n = \sum_{\beta_1 + ... + \beta_k = n, \: \beta_i\ \ge 0} \binom{n}{\beta_1, ..., \beta_m }$$

  \subsection*{A1 $\Diamond$ 5}
  Есть ли у нас коммутативность? Считаются ли одинаковыми одночлены $x_1 \cdot x_2, \: x_2 \cdot x_1$? Будем считать, что являются.

  \subsubsection*{a)}
  Обратимся к задаче A1 $\Diamond$ 9 б.
  Итак, нам надо подобрать такие степени $m_i \ge 0$, что $m_1 + ... + m_n = d$. Это можно сделать $\binom{d + n - 1}{n - 1}$ способами.
  \subsubsection*{б)}
  А теперь давайте посмотрим на наш одночлен полной степени $\le d$ немного иначе:
  $$x_1^{m_1} \cdot x_2^{m_2} \cdot ... \cdot x_n^{m_n} = 1^{d - \sum m_i} \cdot x_1^{m_1} \cdot x_2^{m_2} \cdot ... \cdot x_n^{m_n} :=1^{m_0} \cdot x_2^{m_2} \cdot ... \cdot x_n^{m_n}$$
  Теперь мы считаем количество способов решить уравнение $m_0 + m_1 + ... + m_n = d$. Это $\binom{d + n}{n}$

  \subsection*{A1 $\Diamond$ 6}
  Одним из результатов номера A1 $\Diamond$ 3 является то, что это количество различных слов, которые можно получить перестановками букв из слова \\ $\underbrace{a_1...a_1}_{100 \textup{ раз}}\underbrace{a_2...a_2}_{100\textup{ раз}}...\underbrace{a_{10}...a_{10}}_{100\textup{ раз}},$ а это число является целым.

  \subsection*{A1 $\Diamond$ 7}

  Выберем $k$ элементов первого множества, которые перейдут в 1 элемент. Остальные перейдут в 2. Запишем количество способов сделать это:

  $$\binom{n}{k}$$

  Нам нельзя, чтобы $k$ было равно 0, 1, 4, 5, потому что иначе для какого-то из двух элементов не будет прообраза или он будет только один. Останется

  $$\sum_{k=2}^3 \binom{5}{k} = 2^5 - 2 \cdot(5 + 1) = 32 - 12 = 20$$

  \subsection*{A1 $\Diamond$ 8}

  \subsubsection*{a)}
  $$n^m$$
  \subsubsection*{б)}
  Должно быть $m = n$, тогда это количество перестановок, то есть $m!$
  \subsubsection*{в)}
  Выберем для каждого из наших $m$ элементов $\{1, ..., m\}$ различные образы, чтобы они возрастали. Для этого достаточно выбрать $m$-элементное подмножество $\{1,...n\}$. Это можно сделать $\binom{n}{m}$ способами.
  \subsubsection*{г)}
  Нужно, чтобы $n \ge m$. Выберем $m$-элементное подмножество и биекцию туда. Это $\binom{n}{m}m!$ способов.
  \subsubsection*{д)}
  Это число сочетаний с повторениями $\overline{C}_n^m = \binom{n + m - 1}{m}$
  \subsubsection*{е)}
  Закодируем наше неубывающее отображение строкой из $n + m - 1$ элементов: $n$ нулей, разделенные $m - 1$ единицами -- перегородками. Элементы до первой перегородки перейдут в 1, до второй -- в 2, и т.д. Тогда надо разместить в $n - 1$ возможную позицию $m - 1$ перегородку. Такое отображение будет сюръективным неубывающим. Это можно сделать $\binom{n - 1}{m - 1}$ способами.
  \subsubsection*{ж$^*$)}
  Давайте расмотрим множества отображений $F_i \subseteq \{1..n\}^{\{1..m\}}$, такое, что $\forall f \in F_i\: i \notin E(f)$, где $E(f)$ -- множество значений.
  Заметим, что $|F_i| = (n - 1)^m$, $|F_{j_1} \cap F_{j_2} \cap ... \cap F_{j_k}| = (n - k)^m$, если все индексы $j_i$ различны.
  Обозначим искомое количество как $f(m, n)$. Тогда по формуле включения-исключения:
  $$f(m,\: n) = n^m - \sum^{n - 1}_{i=1} (-1)^{i - 1} \binom{n}{i} (n - i)^m = \sum_{i=0}^{n - 1} (-1)^i \binom{n}{i} (n - i)^m $$


  \subsection*{A1 $\Diamond$ 9}
  \subsubsection*{a)}


  $$\underbrace{(1 +...+ 1) + (1 + ... + 1) + (1 + ... + 1)}_{m\textup{ скобок}} = n$$
  Таким образом, нам нужно для каждой единицы выбрать скобку, а это сюрьективное неубывающее отображение. Таких $\binom{n - 1}{m - 1}$ штук.

  \subsubsection*{б)}
  Сведем к пункту а. Для этого сделаем замену $x'_k = x_k + 1$. Теперь мы решаем в натуральных числах уравнение
  $$x'_1 + ... + x'_m = n + m$$
  У этого уравнения $\binom{n + m - 1}{m - 1}$ решений.

  \subsection*{A1 $\Diamond$ 10}
  \subsubsection*{a)}
  Нетрудно их просто все перечислить:

  \begin{multicols}{2}
  \begin{enumerate}
      \item $(1, 1, 1, 1, 1, 1)$
      \item $(2, 1, 1, 1, 1, 0)$
      \item $(2, 2, 1, 1, 0, 0)$
      \item $(2, 2, 2, 0, 0, 0)$
      \item $(3, 1, 1, 1, 0, 0)$
      \item $(3, 2, 1, 0, 0, 0)$
      \item $(3, 3, 0, 0, 0, 0)$
      \item $(4, 1, 1, 0, 0, 0)$
      \item $(4, 2, 0, 0, 0, 0)$
      \item $(5, 1, 0, 0, 0, 0)$
      \item $(6, 0, 0, 0, 0, 0)$
  \end{enumerate}
  \end{multicols}

  \subsubsection*{б)}
  Аналогично:

  \begin{multicols}{2}
  \begin{enumerate}
      \item $(3, 2, 2)$
      \item $(3, 3, 1)$
      \item $(4, 2, 1)$
      \item $(4, 3, 0)$
      \item $(5, 1, 1)$
      \item $(5, 2, 0)$
      \item $(6, 1, 0)$
      \item $(7, 0, 0)$
  \end{enumerate}
  \end{multicols}


  \subsubsection*{в)}
  К сожалению, сегодня я не обладаю инфорацией о точном значении $p$ и $q$, так что придется искать какое-то другое решение.

  Итак, у нас $p$ строк, в каждой из которых число от $0$ до $q$. То есть нам нужно построить какое-то отбражение из номера строки в $\{0, ..., q\}$, которое было бы еще и невозрастающим (но необязательно сюрьективным). А это практически задача A1 $\Diamond$ 8 д. Итак, ответ $\overline{C}^p_{q + 1}$.

  \subsection*{A1 $\Diamond$ 11}
  Каждый раз мы будем пытаться свести задачу к предыдущим
  \subsubsection*{а)}
  Это очевидно равносильно количеству отображений $\{1,...,7\}\to\{1,...,4\}$.\\ $\left|\{1,...,4\}^{\{1,...,7\}}\right| = 4^7$.
  \subsubsection*{б)}
  Нам нужно распределить по разным чашкам одинаковый сахар. Пусть в кажой чашке $x_i \ge 0$ кусков сахара. Тогда $x_1 + x_2 + x_3 + x_4 = 10$. Нужно найти количество целых неотрицательных решений. Это по A1 $\Diamond$ 9 б $\binom{10 + 4 - 1}{4 - 1} = \binom{13}{3}$.
  \subsubsection*{в)}
  Итак, у нас одинаковый сахар который должен весь оказаться в одинаковых стаканах. Поскольку стаканы одинаковые, нам не важна их перестановка. Давайте сделаем так, чтобы в первом стакане оказалось больше всего сахара, во втором -- не меньше, чем в первом и т.д. Таким образом получим количество диаграмм Юнга веса 10, в которых не более 4 строк. Их 23:
  \begin{multicols}{4}
  \begin{enumerate}
      \item $(3, 3, 2, 2)$
      \item $(3, 3, 3, 1)$
      \item $(4, 2, 2, 2)$
      \item $(4, 3, 2, 1)$
      \item $(4, 3, 3, 0)$
      \item $(4, 4, 1, 1)$
      \item $(4, 4, 2, 0)$
      \item $(5, 2, 2, 1)$
      \item $(5, 3, 1, 1)$
      \item $(5, 3, 2, 0)$
      \item $(5, 4, 1, 0)$
      \item $(5, 5, 0, 0)$
      \item $(6, 2, 1, 1)$
      \item $(6, 2, 2, 0)$
      \item $(6, 3, 1, 0)$
      \item $(6, 4, 0, 0)$
      \item $(7, 1, 1, 1)$
      \item $(7, 2, 1, 0)$
      \item $(7, 3, 0, 0)$
      \item $(8, 1, 1, 0)$
      \item $(8, 2, 0, 0)$
      \item $(9, 1, 0, 0)$
      \item $(10, 0, 0, 0)$
  \end{enumerate}
  \end{multicols}


  \subsubsection*{г)}
  Здесь у нас разные соломинки, каждой из которых надо присвоить какой-то из одинаковых стаканов. Пусть у нас $m$ соломинок, которые надо разложить в $n$ стаканов (но некоторые могут быть пустыми).
  Решим сначала немного другую задачу.

  Пусть нам надо сложить $m$ соломинок в $k$ стаканов, но чтобы ни один стакан не оказался пустым. Это можно сделать $f(m, k) / k!$ способами (см A1 $\Diamond$ 8 ж). То есть мы взяли количество сюрьективных отображений и поделили на количество перестановок стаканов.

  Тогда искомое количество способов:

  $$\sum_{k = 1}^n \frac{f(m, k)}{k!} = \sum_{k = 1}^n \frac{\sum_{i=0}^{k - 1} (-1)^i \binom{k}{i} (k - i)^m}{k!}$$

\end{document}
